% This is an example file showing how images should be inserted into LaTeX documents.
% 
% The following file type rules should be followed as closely as possible.
% Sometimes, logos and other stuff will only be available as PNG, and that's fine.
%
% Allowed file types (anything vector / plaintext):
%  - svg
%  - pdf
%  - eps
%  - ps
%
% Disallowed file types (raster / bitmap graphics):
%  - png
%  - bmp
%  - jpg
%  - gif
%

% \begin block starts a new environment of type figure.
% [H] places the figure very close to the place it's included in the text.
% [H] aka. PLACE HERE
\begin{figure}[H]
% \centering aligns the figure to the middle of the page.
    \centering
% This line includes the actual image, sets the width and assures that the img looks correctly.
% Please avoid writing the file extension. eg "path/pic.svg" should just be "path/pic"
    \includegraphics[width=12cm,keepaspectratio]{images/subject/pdf/filename}
% Every figure needs a caption
    \caption{Very descriptive caption that describes the picture}
% The figure label should be discriptive, and is used for referencing
% \ref{fig:unique_figure_label} returns eg "Figure 1.1"
% figures start with fig:
    \label{fig:unique_figure_label}
% Close the current scope.
\end{figure}

% ============= COPY PASTEABLE BELOW ================= %

\begin{figure}[H]
    \centering
    \includegraphics[width=12cm,keepaspectratio]{images/subject/pdf/filename}
    \caption{Very descriptive caption that describes the picture.}
    \label{fig:PLEASE_FIX_ME}
\end{figure}

% ============= COPY PASTEABLE ABOVE ================= %


