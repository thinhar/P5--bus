\subsection{2. Prototype - Bus stop procedure}
The second component is focused on detecting and performing a bus stop, and is built using the component of the steering object. 

Using the requirements, this component needs to:\\
    - Detect Bus Stops\\
    - Stop parallel to the curb\\
    - Stop within 1 cm of the curb

The track will need to have some bus stops build on it, and it will also need to have some way to signal to the bus that a bus stop is coming up. This will most likely be a coloured piece of tape, that the bus will recognise with a LEGO colour sensor.\info{This is at least the plan as of this moment.}  Furthermore the track will also have to be circular to allow for the bus to keep driving around on the track and make repeated stops at the bus stops.








%OLD:
%As previously mentioned
%\ref{solvingRequirements
%\todo{fix}, we intend wish to detect a physical object emulating a sign and not just a drawing on the tracks. Towards this end we will be attempting to use a NXT CamV4 Sensor and some image recognition algorithm. 


%The bus scale should be a close approximate of a real bus scale as \cite{DriveingCurves}, 