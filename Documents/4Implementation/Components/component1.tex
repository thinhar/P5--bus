\subsection{1. Prototype - Staying Within the Lines}

\subsubsection{Component Requirements}
The requirements for this component are a reduced amount of the total requirements for the project.

From requirement 1, the component must be able to navigate a track constructed with the same relation between its width and the width of the bus as the \todo{Nej er først prototype.2}Danish Road Directorate dictates for Danish roads. This means the bus needs to drive, brake and turn. The bus' maximum turning radius only needs to be at least as much as the maximally allowed curve on Danish roads. 

As a less demanding requirement 4, the first component should avoid collisions with obstacles from the front. \unsure{The second requirement isn't strictly part of the steering "component", instead it's part of the overall bus prototype}

\subsubsection{Prototype Construction}
To create the component focused on steering the bus through the track, which will later be used in the final product, a prototype designed solely for this purpose is built. The prototype needs to be able to drive, turn and brake. Furthermore, it needs to be able to combine these mechanisms to stay within the lines of the track. A LEGO Servo Motor is connected to the rear wheels, which provides forwards and backwards movement capabilities. The speed consistency of the bus is not important in this prototype, however driving at a higher speed allows for testing the capabilities of the sensors and their processing speed.
%however the Servo Motor should be set at \todo{80\%}, to make sure the bus can drive at high speed and still stay within the lines at all times, but the actual speed are not to be measured.

\textbf{Turning}\newline
The front wheels are used for turning and is powered by a LEGO Servo Motor. It should be able to turn the same amount of degrees to both sides, however, there is no requirement for a specific turning angle for this prototype. A thing to note is that the track for the bus should be designed such that the bus can perform the turn within the lines.

\textbf{Obstacle Detection}\newline
To prevent crashes, the prototype supports obstacle detection by having a LEGO ultrasonic sensor mounted at the very front of the bus in a fixed position. The sensor detects obstacles up to 70 cm ahead and 30 degrees to both sides. If it predicts that a collision might occur, the bus stops immediately, overriding both the driving and turning procedures. When the obstacle is no longer a hindrance, it will return to its normal state.  % Later prototype should be able to get out of this situation

\textbf{Follow Track}\newline
For the prototype to stay within the lines of the track, a LEGO NXTCam-v4 should be placed at the front in the center. The sensor should be aimed at the ground in front of the bus in order to see the tracks.

The bus itself should be formed like a long rectangle so that it gets the form of a bus. It should have \todo{1:3?} length to width ratio.

\subsubsection{Track requirements}
The designed track for this prototype can be seen in figure \ref{Track1Layout}. The layout of the track is quite simple and only contain some of the requirements described in section \ref{Requirements}. The reason for this is to only test the features of this prototype and as such, it is designed uniquely for this prototype.

The requirements for the track has therefore been limited to:
\cite{DriveingCurves}
\begin{itemize}
  \item One lane, with a width of the minimal turning space requirement.
  \item 6\% extra lane width on each side.
  \item 180 degrees turn with the minimal length and width.
  \item A Small turn.
  \item Pedestrian crossing for testing obstacle detection.
\end{itemize}

\begin{figure}[H]
    \label{Track1Layout}
    \centering
    \includegraphics[width=0.8\textwidth]{Images/Tracks/Track1.PNG}
    \caption{Track 1 Layout in cm}
\end{figure}

\todo{Insert Final layout}







