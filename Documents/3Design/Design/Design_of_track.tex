\section{Design of the track} \todo{This section needs to be checked for correctness.}

In order to fulfil the requirements of the project a track for the bus will have to be made, but before we can create the track it will first have to be designed.

The track should have two lanes, to mimic the pre-existing infrastructure with bidirectional roads. The track should furthermore have bus stops that the vehicle can drive into, such that passengers can get on/off the bus.
Furthermore the track should be designed to the scale of the bus, such that the turning rate of a corner should fit the maximum turning radius of the bus, and the width of the road should be designed after real danish roads based on the laws given by the danish road directorate \cite{roadRules}\cite{DriveingCurves}. Therefore the track needs be designed such that the bus has the needed space to turn around the corners without needing to reverse.

To help the sensors detect when to switch to a bus stop lane, the track should have special formed/coloured objects placed next to the bus stops, which the sensor(s) can recognise. Furthermore if any people are standing at the bus stop to get onto the bus, they should be placed close to the bus stop sign.

To draw the track that the bus will drive on, black tape should be used such that the sensors that perform line tracking can more easily detect and stay within the lines. Black has been chosen over white, even though white is normally used to draw track lines. But since white can be hard to detect because of light reflection the tracks will use black lines, this is to focus on a more complete bus implementation rather than focusing on tracking the lanes themselves.