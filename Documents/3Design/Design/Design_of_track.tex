\section{Design of the track} \todo{This section needs to be checked for correctness.}

In order to validate that the requirements of the project are met. A track for the bus has to be made, and as such, tracks are designed with the goal of being able to complete the tasks, which are based on the requirements.

The track should have two lanes, as to mimic the pre-existing infrastructure, which usually has bidirectional roads. Furthermore the track should have bus stops that the vehicle can drive into, such that passengers can get on/off the bus.
The track should be designed to the scale of the bus, such that the turning rate of a corner should fit the maximum turning radius of the bus, and the width of the road should be designed after real danish roads based on the laws given by the danish road directorate \cite{roadRules}\cite{DriveingCurves}. As such the track needs be designed such that the bus has the needed space to turn around the corners without needing to reverse.

To help the sensors detect when to switch to a bus stop lane, the track should have special formed/coloured objects placed next to the bus stops, which sensor(s) can recognise. Furthermore if any people are standing at the bus stop to get onto the bus, they should be placed close to the bus stop sign.

To draw the track that the bus will drive on, black tape is used such that the sensors, which perform line tracking can more easily detect lines, and as such be able to stay within the lines. Black has been chosen over white, even though white is normally used to draw track lines. This is because white can be hard to detect due to light reflection. As such the tracks use black lines, as to to focus on a more complete bus implementation rather than focusing on tracking the lines.
