\chapter{Introduction}

Currently, a lot of effort has been put into making self-driving vehicles. To give an idea of the scope of this challenge, some of the functionality required for driving properly and safely are listed below.

The vehicle needs to:
\begin{itemize}
\item Drive to the correct location
\item Get to the location in a timely manner
\item Estimate the optimal driving speed with regards to the type of road and weather
\item Not drive in any way that may cause a crash for itself or other vehicles
\item Drive to a gas station for refuelling prior to running out of gas
\item Diagnose defects
\item Automatically contact the authorities in case of emergency
\end{itemize}

The major benefit of making vehicles autonomous is to increase safety while driving. Most accidents are caused by fault of the driver \cite{baddriver}, whereas self-driving vehicles never get tired and never have a lapse of concentration. Like the automation of any task in society, it also frees up the people that would have otherwise spent time doing that task. Busses can also be made autonomous. These require additional functionality besides the normal challenges that autonomous vehicles face; these are listed below.

A bus needs to:
\begin{itemize}
\item Detect passengers at bus stops that wish to get on
\item Allow passengers to signal that they wish to get off at the next stop
\item Detect when all the passengers have gotten on or off the bus at a stop so it can resume driving
\item Be able to assist passengers in wheelchairs or passengers with strollers in some way to allow these to enter easily
\item Limit its acceleration/deceleration since passengers don't usually wear seat belts and some are often standing upright
\end{itemize}

We will build a small scale LEGO model of an autonomous bus and attempt to face some of these challenges. This will be elaborated upon in the following problem statement. 
