\section{Project Requirements} \label{Requirements}
In this section the project requirements the final product will need to meet in order to solve the problem statement will be descried. 

%In this section we describe the requirements that the final product will need to meet in order to solve the problem statement.%...the requirements for the project will be described. The requirements are meant to be guidelines for the project and the goals which need to be met. The order of appearance is dependant on their importance.

%To test if the bus is able to meet its requirements, test tracks will be constructed. The requirements for the track and the bus will be specified below.
%\info{This section should concern additional elaboration of the chosen requirements from the brainstorm. It should not contain design choices, and it is also doubtful that the requirements should be split into vehicle requirements and track requirements.}

%\subsection{Vehicle Requirements}
%Metatext

\begin{description}
    \item [(1) Follow Track]
    The bus should be able to detect the lines which indicates where the lane is, and stay within these bounds. The size of the track and the LEGO bus should be of similar proportions to the busses driving on danish roads. This also applies to the turning degrees of any road turns. \todo{Første gang læseren hører om dette, mangler et afsnit eller noget, som gav os denne design beslutning}
    
    \item[(2) Stop at Bus Stops]
    The bus should detect, and stop at bus stops on its route if needed, and resume driving afterwards. 

    \item[(3) Switching Lanes] 
    The vehicle should be able to switch lanes. This functionality is useful, both for overtaking other vehicles and switching to dedicated bus lanes. 
    
    \item[(4) Detecting Obstacles]
    Detecting obstacles is a crucial part of preventing crashes and other potential accidents. As such the bus needs to be able to detect obstacles in front of it and perform preventative measures to prevent the collision. In normal busses the response time and field of view of the human driving the bus is the obstacle detection. As such the response time and field of view of our bus should be at least on par with normal human drivers. Seeing as this is highly improbable and unnecessary for a LEGO bus, we will consider this requirement fulfilled as long as it can: detect and react to obstacles in a 60 degree cone to the front within less than a second, and detect and react to obstacles within a 150 degree arch (60 degree cone in the front, and 90 degrees to whichever side the bus is turning) whenever the vehicle is switching lanes. Additionally, we should be able to detect obstacles of a distance between 0 cm. from the bus to the distance it can cover within 2 seconds of driving at max speed \cite{holdAfstand}.

    \item[(5) Stop Button]
    Busses are not required to stop at every bus stop if no potential passenger is to be seen. And therefore need a way to signal which stops the bus should stop at. 
    
    \item[(6) Speed Limiter]
    The bus should keep track of the current speed limit and keep the correct velocity in each speed zone. Specifically, we will consider this requirement fulfilled if it can automatically switch between at least two different speeds, given two different signals/physical signals. \unsure{Skriv tape Color?}
    
    \item[(7) Turn Signal]\unsure{Fjern dette punkt?}
    Signalling that a vehicle wants to turn is very helpful for the other vehicles on the road. The bus should have some way of indicating to the surroundings that it is turning/going to turn.
\end{description}

These requirements are the focus points for this project and every design decision will be valued based on these. Our core requirements for the minimally viable product are 1-5, whereas 6 and 7 are less essential. However we decided to keep 6-7 as part of the requirements even though they are more nice to have then need to have.
%\subsection{Lego vehicle and sensors requirements}
%This subsection will decide upon the requirements for the LEGO bus and the sensors that it uses.

%\subsubsection{Motors}
%The LEGO vehicle should use two servo motors\unsure{is this not a design thing? why is this a requirement}? One for the turning mechanism and one for driving forward and backwards. The turning mechanism should be the front wheels. 
%The LEGO bus needs a number of motors in other to drive itself forwards and backwards and in order to steer.\info{Have changed this subsection to be much less specific.}






% table overview with rankings of both track and car + prototypes