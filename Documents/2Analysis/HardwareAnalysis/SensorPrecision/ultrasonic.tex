\subsubsection{Ultrasonic Sensor} \label{Analysis:ultrasonicTests}
The Ultrasonic Sensor is build with the purpose of recognize objects, avoid obstacles, measure distances, and detect movement. It does this as the name implies, sending sound waves and calculating the time it takes for that sound wave to hit an object and return back to the sensor by knowing the sound speed.

In figure \ref{NXT-Ultrasonic-Sensor} the ultrasonic sensor can be seen. It consists of two "eyes", a transmitter(left) and a receiver(right)\cite{ExposedLEGOUltrasonic}. 

\begin{figure}[H]
    \centering
    \includegraphics[width=0.5\textwidth]{Images/Analysis/NXT-Ultrasonic-Sensor.png}
    \caption{The ultrasonic sensor: Left eye is the transmitter, and right eye is the receiver.}
    \label{fig:NXT-Ultrasonic-Sensor}
\end{figure}

% Billede http://www.legoengineering.com/wp-content/uploads/2013/04/NXT-Ultrasonic-Sensor.png

The Ultrasonic Sensor measures the distance in centimeters and
inches. The sensor range is 0 to 255 centimeters with a precision of +/-3 cm. But this is with the perfect shape objects, the real range depends on the shape of the object it tries to detect. For example, soft and curved objects like a ball can be hard to detect. While large-sized hard surface  objects like a wall, can be easier to detect\cite{}. Furthermore if the range is 255 it's an indicator the sound wave didn't return in time, and didn't detect any objects.
% https://www.generationrobots.com/media/Lego-Mindstorms-NXT-Education-Kit.pdf




A sound beep does not travel in a narrow straight line. Instead it spreads out like the light beam of a torchlight. As a result the sensor not only detects objects that are exactly in front of the sensor, it also detects objects that are somewhat to the left or right.







result of tests???



The sensor testing chapter can be found in appendix



\ref{ultrasonicSensorTesting}\todo{Add ref to camerasensortesting}.
