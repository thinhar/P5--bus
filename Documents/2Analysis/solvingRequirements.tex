\section{Solving the requirements} \label{solvingRequirements}
In this section, we plan how we intend to solve the requirements that have ambiguous elements. Additionally, we specify the assumptions that we are going to make for the sake of simplicity. \info{This section mentions a lot about how the different requirements should be solved. But isn't deciding how the different requirements should be solved a design issue, and thus belong to the design chapter. In this analysis chapter, it is fine to take a more in-depth look at the requirements, but they should not be designed.}

\begin{description}
    \item [(1) Follow Track]
    We assume perfectly consistent road markings with no imperfections and unintentional holes. We are planning to emulate road markings using black tape. This necessitates that we use some sensor that can detect the colour or light difference between the tape and the rest of the track. These results should be the same regardless of any minor light level differences in the room, as long the room is still lit up akin to daylight. 

    \item[(2) Detecting Bus Stops]
    We do not intend to solve this requirement for any type of bus stop design; instead, we will select such a design and create a component to detect bus stops. While the design could simply be a mark on the tracks (which could reuse the technology used for requirement 1), however, this does not mimic real-life bus stops. Instead, we will be using physical signs to indicate bus stops. We will attempt to detect these stops using a camera and a corresponding algorithm to perform the recognition. 
    
    As for parking the bus at the stops, we intend for the bus to park parallel to the bus stop, with a maximum distance of 1 cm to the edge at any point. Note that the bus is intended to stop at any bus stop, whether passengers are waiting or not, as detecting passengers wishing to enter is a complexity that this project does not intend to handle. 

    \item[(3) Switching Lanes] 
    We assume that at any point lane switching is allowed, lines will be replaced with dotted lines. 

    \item[(4) Detecting Obstacles]
    We intend to detect obstacles using a single sensor, which turns on whenever the bus is instructed to switch lanes, as to verify that no obstacles are found in the other lane. If this does not work, multiple sensors may be needed.

    The sensor should detect obstacles within a 60° angle and a distance between 0 cm. from the bus and the distance it can cover within 2 seconds of driving at max speed \cite{holdAfstand}.
    
    \item[(5) Stop Button]
    Although this fits less well with the reality, we will not be using a physical stop button inside the bus, given that it will be tough to activate. Instead, we will use something that can be activated remotely, which will cause the bus to halt at the next stop.

    \item[(6) Speed Limiter]
    To solve this requirement we intend to reuse either the technology used to follow the track (requirement 1) or the technology for image recognition (requirement 2). We will use this to read either marker on the tracks or a sign that signifies a certain speed limit. We intend to measure the speed of the bus using an accelerometer. 
    
    \unsure{Change this last line if we ever remove the accelerometer section.} % I think this one makes absolutely no sense -Jakob

    \item[(7) Indicator Signals]
    To indicate to others when the bus is performing a turning manoeuvre we would add lights to both sides of the bus.
    
    \todo{Do we really want this requirement? We know this isn't going to work, because we don't have enough slots in our NXT to signal properly.}
    
\end{description}


%\section{Requirement Clarification and Scope Limitation}
%While the previously mentioned requirements are nice goals, some of them are either much too difficult to solve fully or not directly actionable in their current state. In this section we specify precisely what we consider to be necessary for each of the previous requirements to be considered fully achieved.