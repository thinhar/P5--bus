\section{Uppaal}

Uppaal is a powerful tool that allows for the user to model, validate and verify real-time systems. Uppaal has been made in a collaboration between the department of information technology in Uppsala university, and the department of computer science at Aalborg university.

Uppaal consists of 3 elements. There is a description language, there is a simulator, and lastly there is a model-checker.\\
The purpose of the description language is to serve as a way to describe system behavior in the form of networks of automata. It is a non-deterministic guarded command language which have data types.\\
The simulator is used as a validation tool. It can examine the possible dynamic executions of the system in the early design.\\
The model-checker is the last element of Uppal, and it covers the exhaustive dynamic behavior of the system. \cite{uppaal}

Uppaal is used to check our model, and detect the possible faults in our system. 