\section{Discussion}


Dependability??


\subsection{Consequences of missing colour sensor}\todo{Is this descriptive enough?}

We wanted to use a colour sensor to implement some of our functionality(speed zone, bus stop). Initially, we had gotten a colour sensor that according to tests, performed at the start of the project period, worked fine. However, when we wanted to use the colour sensor as a part of the implementation it no longer worked, and it was not possible to receive another colour sensor instead.
The colour sensor was meant to be attached underneath the bus where it could read the colour of whatever the bus drove over. One colour would signal that it was entering a speed zone, and another would signal that a bus stop was coming up. When it entered the speed zone, a speed limit would be set for the bus, such that it never exceeded a set speed. When it was noticed a bus stop was coming up it was supposed to start a procedure to pull over into the bus stop and stand still for a little while to allow for simulating passengers getting on board.
Both of the two tasks would be using the colour sensor however they are not both able to get data from the colour sensor at the same time since we only have a single processor. The solution to this issue would be to have a single task that puts the colour data into a buffer. The two tasks could then look into this buffer to determine what the colour is. 


\subsection{NXTCam Sensor}
NxtCam hijackede lidt projektet
Det blev til NxtCamProjektet i stedet for at være om en bus der bruger en sensor
Det handlede altid om: "Hvordan bruger vi dataet fra NxtCam til noget nyttigt?"

Derudover: som benævnt i \ref{noteworthyCam}, havde vi en del problemer med sensoren og mystiske fejl, der var svære at pinpointe, fikse og arbejde uden om. Det er lidt grunden til at vi vores NXTCamSensorController \todo{Hedder den det? Tjek det endelige program} ikke abstraherer fuldt fra sensornære elementer, men at algoritmen også skal tage hensyn til ødelagt data. 

We could have used less fancy sensors and in stead focused our efforts on coding logic that could make the bus drive properly. Instead, we used most of our effort on understanding what the NXTCam returned and why. 

\subsection{Implementation does not fully fit Design}
The functionality of the bus that we prioritised the highest and spent by far the most time on was for it to drive within a lane by use of the NXTCamSensor. So much so, that we started experimenting and tinkering with it prior to the design being finished. 

\subsection{The Driving Algorithm}
Is it optimal? It pingpongs between the road markings. 

\subsection{Camera Placement}
Something we figured out too late was that the camera should have probably been placed lower. Its angle of view should have been tweaked as well to allow for less lookahead, but still enough knowledge about the lines. Problem is that during a turn, we lose sight of the line beacuse it's within the 8 \todo{8?? Look at mathcad calcs} cm. between the sensor and the bus, that the sensor cannot detect. It would have also made the camera more stable, which could have made measurements slightly more reliable. We did not need as large of a lookahead as what we ended with (42 cm? \todo{Look at mathcad calcs}). 